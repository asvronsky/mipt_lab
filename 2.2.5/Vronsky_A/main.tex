\documentclass[a4paper,12pt]{article}
\usepackage[a4paper,top=1.3cm,bottom=2cm,left=1.5cm,right=1.5cm,marginparwidth=0.75cm]{geometry}
%%% Работа с русским языком
\usepackage{cmap}					% поиск в PDF
\usepackage{mathtext}
\usepackage[T2A]{fontenc}
\usepackage[utf8]{inputenc}
\usepackage[english, russian]{babel}
\usepackage{amsmath}
\usepackage{amssymb}
\usepackage{latexsym}
\usepackage{physics}
\usepackage{multirow}
\usepackage{longtable}
\usepackage{physics}
\usepackage{xfrac}

%%% Нормальное размещение таблиц (писать [H] в окружении таблицы)
\usepackage{float}
% \restylefloat{table}

\usepackage{graphicx}
\graphicspath{ {./figures}{./processor_output}}

\usepackage{wrapfig}
\usepackage{tabularx}

\usepackage{pgfplots}
\pgfplotsset{compat=1.8}
\def\mathdefault#1{#1}
\everymath=\expandafter{\the\everymath\displaystyle}

%%% Дополнительная работа с математикой
\usepackage{amsmath,amsfonts,amssymb,amsthm,mathtools} % AMS
\usepackage{icomma} % "Умная" запятая: $0,2$ --- число, $0, 2$ --- перечисление

%% Номера формул
\mathtoolsset{showonlyrefs=true} % Показывать номера только у тех формул, на которые есть \eqref{} в тексте.

%% Шрифты
\usepackage{euscript}	 % Шрифт Евклид
\usepackage{mathrsfs} % Красивый матшрифт

%% Свои команды
\DeclareMathOperator{\sgn}{\mathop{sgn}}

%% Перенос знаков в формулах (по Львовскому)
\newcommand*{\hm}[1]{#1\nobreak\discretionary{}
	{\hbox{$\mathsurround=0pt #1$}}{}}

\date{\today}

\usepackage{textcomp}
\usepackage{gensymb}

\usepackage{indentfirst} % Русская традиция первого отступа
\usepackage[inline]{enumitem} % работа со списками

%% Импорт csv таблиц
\usepackage{booktabs}
\usepackage{colortbl}
\usepackage{pgf}
\usepackage{pgfplotstable}
\pgfplotstableset{col sep=comma, search path=./processor_output}
\pgfkeys{/pgf/number format/1000 sep={}}
\newcolumntype{g}{>{\columncolor[gray]{.8}}c}

% проверяет наличие аргумента в подстроке
\usepackage{xstring}
\newcommand\ifin[2]{\IfSubStr{,#2,}{,#1,}}

\newcommand*{\nth}[2]{%
  \foreach \x [count=\k] in #1{%
        \ifnum\k=#2{%
          \x
        }\fi%
    }
}

\usepackage{caption}
\usepackage{subcaption}

\newcommand\thpar[3]{\left(\frac{\partial #1}{\partial #2}\right)_{#3}}
\newcommand\mujt{\mu_{\text{ДТ}}}

\begin{document}

\begin{titlepage}
	\begin{center}
		{\large МОСКОВСКИЙ ФИЗИКО-ТЕХНИЧЕСКИЙ ИНСТИТУТ (НАЦИОНАЛЬНЫЙ ИССЛЕДОВАТЕЛЬСКИЙ УНИВЕРСИТЕТ)}
	\end{center}
	\begin{center}
		{\large Физтех-школа прикладной математики и информатики}
	\end{center}


	\vspace{4.5cm}
	{\huge
		\begin{center}
			{\bf Отчёт о выполнении лабораторной работы 2.2.5}\\
      Определение вязкости жидкости по скорости истечения через капилляр
		\end{center}
	}
	\vspace{2cm}
	\begin{flushright}
		{\LARGE Выполнил:\\ Вронский Александр Сергеевич \\
			\vspace{0.2cm}
			Б05-226}
	\end{flushright}
	\vspace{8cm}
	\begin{center}
		Долгопрудный\\
		\today
	\end{center}
\end{titlepage}

\textbf{Цель работы:} 1) определение вязкости воды по измерению объёма жидкости, протекшей через капилляр; 2) определение вязкости других жидкостей путём сравнения скорости их перетекания со скоростью перетекания воды.

\textbf{В работе используются:} сосуд Мариотта; капиллярная трубка, мензурка; секундомер; стакан; микроскоп на стойке;

\section{Теоретические сведения}

\end{document}
